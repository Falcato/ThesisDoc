%!TEX root = ../dissertation.tex

\begin{otherlanguage}{english}
\begin{abstract}
% Set the page style to show the page number
\thispagestyle{plain}
\abstractEnglishPageNumber

Ad hoc networks are emerging as a possible solution to offload traffic from the infrastructure network, or to help disseminate messages when the infrastructure network is not accessible. However, most existing frameworks only provide single-hop or broadcast message transfers, limiting the possible uses of the ad hoc network. This thesis proposes a detailed framework, with a routing protocol, which is able to establish an ad hoc network allowing users to exchange data in a unicast, multi-hop manner. The developed framework can be used in Android applications for different purposes. It currently establishes Bluetooth connections, but is easily adapted to use other communication technologies. Then, a prototype peer-to-peer Android application is developed, based on the proposed framework, which enables users to real-time web browse without an active Internet connection, where web pages may be exchanged in a multi-hop manner. The prototype was tested in experimental settings. The results show that the proposed framework and developed application provide an acceptable quality of service, that could be further increased using certain technological advances.

% Keywords
\begin{flushleft}

\keywords{Ad hoc network; Bluetooth; Wi-Fi Direct; Android; Peer-to-peer application}

\end{flushleft}

\end{abstract}
\end{otherlanguage}