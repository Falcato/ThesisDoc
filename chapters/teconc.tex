\subsection{Conclusion}

Mobile network technologies have been improving at a fast pace, since the demand for higher speeds is constant. With the evolution of modulation methods to higher-orders and \gls{MAC} protocols improving spectrum efficiency and number of users in the network without interference, the demand for higher speeds has been successfully answered. 5G networks should focus further on resource optimization and a massively distributed \gls{MIMO} system.

Wi-Fi technology has many different standards, some being the natural evolution of the others, some serving different purposes, such as IEEE 802.11s. The Wi-Fi infrastructure mode has been constantly updated with better \gls{MAC} and modulation techniques, allowing for higher data rates and more users on the network. Ad hoc networks have also been evolving being the best candidate to offload some of the traffic in the infrastructure mode, also to achieve smaller, independent networks. Wi-Fi Direct has appeared as a possible method to implement ad hoc networks. Its support is still limited in devices, only allowing for some of the features it can provide. Progresses must be made in order to utilize this technology to its full capabilities.

Bluetooth has had a similar development to the IEEE 802.11 standard, evolving to faster data rates from version to version. Used for smaller networks than Wi-Fi, Bluetooth is widely used to deploy \glspl{WPAN}, now with the concurrence of Wi-Fi Direct, although they can both be used simultaneously. \gls{BLE} was also a big development in low energy networks, allowing for fast data transfer with low power consumption. Each Bluetooth technology has its utility, and the future focus should be in expanding the number of allowed users and range of the networks.

Finally, \gls{NFC} provides technology for yet another type of network. This time with a range even smaller than \glspl{WPAN}. It has a lot of applications but it's widely known for its easy and secure usability in transactions. It is being researched by industry giants like Amazon and AliBaba, but there is still much room for improvement, in security, network range, data rates, \textit{etc.}.

Overall we can say that most technologies have met huge improvements in short periods of time, and the tendency is to continue that way. Data rates will continue to grow as higher-level modulation techniques are discovered and new \gls{MAC} protocols are proposed. More emphasis has been given to smaller device to device networks in later years, has a way to take some load from the infrastructure, or even to for networks relevant to day to day tasks (\gls{IoT}).







